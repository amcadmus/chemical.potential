\documentclass[12pt]{article}
\usepackage{german}
\usepackage{a4}
\usepackage{graphicx}
\usepackage{pslatex}
\usepackage{color}
\pagestyle{empty}
\parindent0cm
% \addtolength{\textheight}{2cm}
\renewcommand{\sfdefault}{cmss}
\renewcommand{\familydefault}{\sfdefault}
\newcommand{\recheck}[1]{{\color{red} #1}}

\newcommand{\kopf}{\noindent
\raisebox{2.5cm}[0cm][0cm]
{
\parbox[c]{0.88\textwidth}
{
\begin{center}
\textsc{\large Institute for Mathematics, Freie Universit\"{a}t Berlin}\\
\textsc{ PD Dr. Luigi Delle Site}
\end{center}
}
}}
%\raisebox{2cm}[0cm][0cm]{\rule{\textwidth}{0.2mm}}}
\begin{document}
\kopf
\vspace{2cm}
\hfill\raisebox{1.5cm}[0cm][0cm]
{\parbox[t]{0.34\textwidth}
{e-mail:\\
{\small luigi.dellesite@fu-berlin.de}\\
Arnimallee 6\\
D-14195 Berlin\\
Tel:\ +49 (0)30 838 75775 \\
Fax:\ + 49 (0)30 838 75412 \\
}}
\vspace{3cm}
\noindent\raisebox{1.5cm}[0cm][0cm]
{\fbox{\parbox[t]{0.6\textwidth}
{Editor of {\it The Journal of Chemical Physics}\\
}}}
\vspace{1cm}

\textbf{Resubmission of the paper: ``Chemical potential of liquids and mixtures via Adaptive Resolution Simulation''}

\vspace{1cm}
\parskip 2ex
Dear Editor

We would like to thank both referees for the criticisms and suggestions.
Below please find a reply to each of the concerns raised by the referees and the
corresponding modification of the manuscript (where required).
In order to facilitate the work of the referees we have highlighted in red, the additions/modifications in the manuscript.

\vskip 1cm
Sincerely,

Luigi Delle Site

\newpage

{\bf For the Editor only:}


Regarding the comment of {\bf Reviewer 1}:\\
{\color{blue} \it  It is not easy to judge the relevance of this
  work because it might contain enough simulations, but not new information. In fact, its main contribution is a table, with comparison with reported data.}

In the previous paper published about one year ago in Physical Review X,
we have developed what we have {\bf openly} proposed in that paper as a {\bf rigorous approach} where all the formulas were derived analytically and checked mathematically.
The whole theory is based on two assumptions:

(1) Thermodynamic limit for the {atomistic and the coarse-grained region. (i.e. the number of molecules in both of the two
  regions are infinitely large, and the coarse-grained region is infinitely larger than the atomistic region)}.
% (1) Thermodynamic limit for the coarse-grained region (i.e. number of molecule very large).

(2) Hybrid region considered as a thin filter compared to the size of the two other regions.

In a simulation of course these conditions cannot be met at the ideal level, thus the original idea of this paper was that of treating real systems in order to explore the numerical {\bf accuracy} that the formulas can provide. Thirteen systems (about 300 simulations) were studied with both AdResS and with the thermodynamic integration, this {\it per se} is a major numerical effort, even for large computer facilities, however, the systems were carefully chosen as representative of different classes of liquids and mixtures characterized by different physics. The {\bf new information} consists of thirteen values of the chemical potential which are reproduced by GC-AdResS, thus showing that the method is trustable for such kind of calculations and that the hypothesis (1) and (2) above can be easily met in standard simulations.
In the paper, we have made clear the intention described above, but it seems that the first referee has missed it.
Moreover, the request of the first referee about showing concentration dependent behaviors in order to show that we are consistent with previous work at very low concentrations was actually not of major interest, since we always compare each GC-AdResS calculation with the corresponding TI at the same concentrations. {\bf In our view this was clear also to the referee and it was also clear to him/her the amount of additional computational effort he/she was asking for}.

However, independently from the referee's comment, we decided to do the study about the concentration dependent. In fact, in case of positive outcome, we would show that our method is very robust over different thermodynamic state points (and, as minor point, we would make happy the referee about the consistency with previous work). Indeed our results strongly support such a conclusion. However the computational cost was massive (about 300 additional simulations!), for this reason we ask you to consider this point in case the referee requires additional simulations.

In addition, we did not like the comment of the referee:\\
{\color{blue}\it ``There is also a theoretical
  weakness in the sense that the main equation used to evaluate the chem. potential seems to be derived in an ad hoc fashion, and ``proven numerically'', as the authors wrote.}

It was not the aim of the paper to have an analytic derivation of the formula, this does not mean that the approach is weak and above all the referee gave the impression that we obtained something by manipulating formulas and computational tools in order to ``force'' our results.

Given our major effort over the last 2-3 years to merge analytic/mathematical work with numerical applications to realistic chemical systems, the opinion of the referee is quite discouraging. 
However, our impression is that the first referee did not really read our previous paper and/or he/she may not have the background for catching analytic/mathematical derivations of formulas in full. In any case, in this revised version we have added the analytic derivation of the formula, (and in order to address the point raised by the second referee, we have added also the analytic derivation of the extension of the formula to multicomponent component systems). We have tried to be not too much mathematical, in order to reach the whole community interested in the subject. However, to merge computational results about chemical systems and analytic derivation of the formulas in one coherent paper is not easy, we made a strong effort in such direction and we hope that you will appreciate it.

We have also added two very recent references related to the use of AdResS with the MARTINI force field. The senior authors of such publications pointed out to us the importance of calculating the chemical potential as a test of thermodynamical consistency of the whole system, especially for systems of the kind they are treating (large bio-systems).

Finally, the referee asked us to {\it ameliorate} the language; we are fully aware of not being mother tongue Oxford academics, but we asked several colleagues to read the paper and tell us if it was understandable; we got only positive comments, we cannot do more than that.

We have added all the calculations/information asked by the referees and we are confident that now the paper is ready for publication.


\newpage

{\bf Reply to Reviewer 1:}

{\color{blue}{\it ``..... The idea of calculating excess
    chemical potentials from adaptive resolution techniques is appealing}}

We are pleased that the referee recognizes that the overall idea is interesting; we take this comment as a positive basis of our work.

{\color{blue}{\it ``This work is in fact a sort of extension of Ref. 5, and its contribution is just to offer a check of the
statements, hypothesis and calculations made on that previous paper. It is not easy to judge the relevance of this
work because it might contain enough simulations, but not new information. In fact, its main contribution is a table, with comparison with reported data.}}

In Ref.5, we claim to have developed a rigorous approach from the
point of view of physics and went even further by exploring how far
one could make it mathematically rigorous/exact. We have found that
under the hypothesis:

(1) Thermodynamic limit for the {atomistic and the coarse-grained region. (i.e. the number of molecules in both of the two
  regions are infinitely large, and the coarse-grained region is infinitely larger than the atomistic region)}.

(2) Hybrid region considered as a thin filter compared to the size of the two other regions.

The method is mathematically rigorous and for some aspects even
exact. Conditions (1) and (2) are ideal, so it was important to know
how they can be met in {practical simulations, especially in
the mixture systems that the solute concentration is dilute.} The aim of this paper is: {\bf to
  explore the numerical accuracy of the theory (for the chemical
  potential) proposed in Ref.5}.  We have done it for thirteen
representative and computationally expensive systems.  We hope the
referee can recognize that this effort is a non-trivial one. However
as asked by him/her, we have added all the simulations requested (see
below).

{\color{blue}{\it On the other hand, the paper lacks important pieces of information and many incomplete statements and definitions (see below). There is also a theoretical
    weakness in the sense that the main equation used to evaluate the chem. potential seems to be derived in an ad hoc fashion, and ``proven numerically'', as the authors wrote. It is not easy to rely on an equation/method which lacks a solid theoretical background (mostly in statistical physics)}}

% \recheck{
%   The main purpose of the paper was to show the numerical accuracy of the theory, so we did not provide the theoretical derivation of
% }
% We have taken such a comment as a motivation to make an effort and finalize a derivation of the formula defined {\it ``weak''} by the referee. The derivation is presented as one of the results of this paper and has got its own section.
We have taken such a comment as a motivation to {provide the} derivation of the formula defined {\it ``weak''} by the referee. The derivation is presented as one of the results of this paper and has got its own section.
Although developed before the submission of this paper, this result was not presented, because the paper was intended to investigate only the numerical accuracy.


{\color{blue}{\it (1) A description of the cg model is lacking. Please specify how do you treat and the interaction potential for
cg molecules of solvent and solute. I suppose that both types of molecules (solvent and solute) free to cross the hybrid layer.
Please confirm. Please add a section about the cg model.}}

We added the procedure in the appendix. It must be noticed that, as a
major result of Ref.5, we actually do not need to derive a detailed cg
model. In fact our approach allows us to use simple spherical models
whose size is about the vdW radius of the specific molecule. The spheres
interact via the simple WCA potentials. It is then the thermodynamic force
that makes the thermodynamics of the cg models compatible with the
atomistic one. Both types of molecules change their resolution when they pass from
one region to another.

{\color{blue}{\it (2) Definitions of $V_{AT}$ and $V_{CG}$ are also lacking in terms of particle indexes. The model definition is important to distinguish between total energies and pair-wise terms, e.g. $V_AT(xij)$.}}

Done. See the new Section IV for a detailed description.

{\color{blue}{\it (3) Figure 1. ``allows to write an exact Hamiltonian for the atomistic region and thus treat the system in a GC fashion.'' This phrase is
difficult to understand. The AT region is connected with the CG domain (via a transition layer which is not small, but much thicker than the
molecular size). Thus either one has a complete Hamiltonian for the whole AT+HYB+CG system, or else, there is no Hamiltonian. In other words, if
the interactions between AT and HYB, and CG molecules are not derived from H, then there is no Hamiltonian. Also, ``...treat the system in a GC
fashion'' is also difficult to understand (maybe the authors mean ``...consider the AT region as an open subsystem''). I would suggest
deleting this phrase and refer in the text to their theoretical findings in Ref. 5 about particle number distributions p(N),etc.}}

Actually this is the very core of our derivation. If the referee think that this is wrong, then Ref.5 is entirely wrong.

In fact we have extended the hybrid region adding a region equal to
the interaction cut off where the resolution is full atomistic, but
such a region is not considered part of the full atomistic region. As
a consequence all the molecules of the atomistic region will interact
with the hybrid region {\bf ONLY} at atomistic level (coarse-grained
molecules are in the region beyond the cut off one). In fact beyond
the cut off radius there is no interaction or better the interaction
is zero. As a consequence the molecules of the atomistic region have a
well defined Hamiltonian: The Hamiltonian of an open system.
It is as we take a subsystem of a full atomistic simulation; the Hamiltonian of the subsystem is well defined, without any artificial interpolation.
Moreover an open subsystem of a much larger system is naturally Grand Canonical.
This is clearly reported in Ref.5 and Eq.12 provides the exact definition of the Hamiltonian.
We have added the exact reference for the equation with the specific Hamiltonian, and we do not believe it should be explicitly reported and discussed in the current paper because it would be redundant and anyway it is not crucial to any of the derivations of this work.


{\color{blue}{\it (4) Page 4. ``...reproduce the first order of the
    probability...'' and ``Higher orders...'' Could the authors
    specify what (small) quantity the order is referred to?, I think
    it could be orders in the density of solute molecules, or maybe in
    density variation with respect the average. Please be more clear
    when referring to ``orders'' elsewhere in the manuscript.}}

This is clearly described in Ref.5; we have defined the first order of
the probability distribution of a system as its molecular density, a
function in three-dimensions. The second order is the two-body
distribution (6-dimensional function) of which the radial distribution
function is a specific expression; the third order is the three-body
distribution (in 9 dimensions) etc etc. The full probability is a
3N-dimensional function of which the different orders are a lower
dimensional approximation.

We have added the explicit definition in the revised version of the paper.

{\color{blue}{\it (5) Page 5. Auxiliary Hamiltonian. For the sake of
    clarity, the authors should write down this Hamiltonian they refer
    to, with full index details. Also, please write down the form of
    thermodynamic force in the auxiliary Hamiltonian. They mention
    that simulations using the auxiliary Hamiltonian were made with no
    thermostat. Did the authors check that the ``energy'' of this
    Hamiltonian is conserved? Please comment on this important
    issue.}}

As before, it is reported in all details in Ref.5 and we thought it
was redundant, but we have added it in the revised version. The
thermodynamic force must have exactly the same numerical formula of
the standard AdResS, otherwise it would not work for our purposes: if
two expressions are formally identical but one has the effect of the
thermostat and the other not then the difference isolates the effect
of the thermostat. For the Hamiltonian system we use the same formula
and impose artificially the same density of the AdResS system thus we
get a condition on the gradient of the molecular density in the
transition region, this is now reported in the paper.

Yes, we have checked the conservation of the energy and added a
section dedicated to this issue in the Appendix. There we report the
advantage and limitations of such an approach (i.e. when we can
consider pure conservation and when the coupling to a thermostat is
anyway needed).


{\color{blue}{\it (6) ``Keeps the density of particles across the system as in AdResS''. Do you mean at constant density?}}

Yes


{\color{blue}{\it (7) ``It must be noticed that the relation we
    obtained form the chem. pot. is similar to that obtained
    in... Ref[6]''. This statement is not clear. The equation this
    method proposes seems not similar to that of Ref. 6. In particular
    the term is quite different from which is the corresponding free
    energy difference appearing in Ref. 6. The authors should
    explicitly write out this essential term of the calculation,
    indicating summations with particle indexes. It would seem from
    the (not-written auxiliary Hamiltonian) that the particles index
    of the first ``w'' is not the same as that of the ``nabla
    w''. Please derive and clarify this term so anybody could
    understand it.

    The effect of discrepancies with Ref.6 are not
    simple to trace, but certainly not zero. The authors state:
    ``H-Adress and CG-Adress are the same up to first order'' (please
    again, in what?). A reference proving this statement is not
    mentioned, but I guess the authors refer to Ref. 7 for that
    prove. However after reading that paper (Ref. 7) I saw no prove of
    this (just essentially the same phrase, without prove). The
    authors should prove this statement here or delete this phrase, or
    comment about the lack of comparisons, at present.}}

We see the point of the referee, perhaps we should do a paper with explicit formulas and data to show similarity and differences between the two methods.
However since this is not a relevant part of the paper, we decided to remove any reference to similarities and differences with Ref.6.
and add the reference from the same authors on mixtures.


{\color{blue}{\it (8) Fig. 2. Please indicate the locations of the HYB
    domain. A difference of about 20 percent in density is not small,
    although *inside* the AT domain, the g(r) might not be too
    sensible to such density difference (as shown in the fig.). On the
    other hand, contrary with what the authors state below, a large
    transition region is beneficial, because molecules have larger
    space to adapt to the change in resolution. Please correct this.}}


{The Fig.2 (Fig.4 in the revision) plots the density profile only for the HYB region, as the caption states.}
% We disagree.
{The relatively large deviation is due to the sampling error, because the
  number of TBA molecules in the system is very small (80 for mole-fraction 0.02, see the Table II, i.e.~the Table IV in the revision).
  We could reach a much better accuracy of the density if we do longer simulations,
  but it would require much more computer time.}
% We could reach a much higher accuracy if we set the level
% of accuracy of the thermodynamic force higher (but requires more
% computer time).
However, we wanted to show that even with 20 percent of
(maximum) difference we get the chemical potential accurate and the
density in the AT region flat.

Regarding the point that a larger
region is beneficial, we must disagree. This is based on arguments
(which we will give next) and long experience with such systems. What
is important in AdResS is that the AT and CG region are large, in fact
if they are large they are going to thermalize and in general
equilibrate substantially the hybrid region.  We have recently made
tests on related issues and found that if one uses a large thermalized
reservoir then small regions where the density or the the temperature
are not targeted, are automatically equilibrated by the large
reservoir. Moreover our analytic derivation suggests the same.  On the
contrary a HYB region is characterized by non-physical description of
the potentials and thus it does not have as a reference any
thermodynamic state point. As a consequence the larger the HYB region
the larger the perturbation. Of course there is a minimal size that is
a region equal to the interaction cut off.  These aspects have been
explored and described in details in several previous papers among
which:
\begin{itemize}
\item (1) Matej Praprotnik, Luigi Delle Site, and Kurt Kremer, .J.Chem.Phys. 123, 224106 (2005).
\item (2)Silvina Matysiak, Cecilia Clementi, Matej Praprotnik, Kurt Kremer and Luigi Delle Site, J.Chem.Phys. 128, 024503 (2008).
\item (3) S.Poblete, M.Praprotnik, K.Kremer and L.Delle Site,  J.Chem.Phys. 132, 114101 (2010).
\end{itemize}


{\color{blue}{\it (9) Table I. What results of Ref. 15 corresponds to
    table I?, this is not mentioned. Results in Ref 12 and 15 were
    made at lower concentration, but the authors state that the excess
    chem pot is not sensible to the solute concentration in this
    regime. And say that ``we have verified that such consistency
    holds''. This is important though, and the authors should include
    a graph with the calculated excess chem. pot. against the solute
    concentration, showing that it goes to a plateau at low
    concentration consistent with the experimental value reported (at
    least for some of the cases considered). Error bars are crucial,
    but means nothing without the simulation times required to obtain
    them. The authors should include a more clear measure of this. A
    possible one could be showing the accumulated standard error
    against the simulation time and compare the IPM, the TI (free
    energy estimate using Bennet's method) and their own method. With
    the specification of the system size (table II) this already gives
    an idea of the computational costs.}}

We have added a plot (for one system) of $\mu^{ex}$
v.s. concentration, we have shown that the chemical potential
converges to the experimental value for very dilute
concentrations. Moreover our plot is consistent with that of Lee and
van der Vegt, (ref.18) which is supposed to be {\it ``state of the
  art''} result in the field.  For computational costs, see below.


{\color{blue}{\it (10) Sec. IV. ...a critical appraisal. This comment
    is connected with 8). In this section the authors state that their
    method is not as fast as TI if dilute systems are considered. I
    would just like to comment that this observation seems to conflict
    with the comment made in Ref. 7 (``orders of magnitude faster than
    IPM''). Indeed, Widom method is not very good for large molecules
    in liquids, this is well known. So, as far I understand, the
    author's conclusion is that the best option is to use TI for
    dilute systems while adaptive resolution pays off when dealing
    with many solute molecules. This observation is important and
    should be shown in a graph (e.g. standard error vs. simulation
    time for TI and CG-Adress, in the dilute case and less diluted
    case)}}

We have added a table where we show the total time needed to calculate
$\mu^{ex}$ with GC-AdResS and with TI.  The table show that the total time
needed for obtaining the plot of $\mu^{ex}$ v.s. concentration is
slightly shorter with our method than with TI.  Moreover one must add
the fact that we use relatively small system (5000 molecules) and that
GC-AdResS is not yet optimized in Gromacs.  We hope that this is a
good argument to convince the referee that maybe it is worth to look
at GC-AdResS as a potential complementary (not substitutive) method
for calculating $\mu^{ex}$.

{\color{blue}{\it (11) Recent works: an extension to the Hamiltonian
    based method the authors refer to, appeared in
    ``Phys. Rev. Lett. 2013, 111, 060601''. This work should be cited
    because it precisely deal with mixtures being in close relation
    with this work.}}

Done.


\newpage

{\bf Reply to Reviewer 2}:

{\color{blue}{\it I would recommend that the authors consider the
    following minor revisions. Most significantly, it would be very
    helpful for the authors to discuss the apparently systematic
    errors that arise in their estimation of the chemical
    potentials. The following provides further details and minor
    comments. Table 1 compares the chemical potentials determined from
    GC-AdResS and thermodynamic integration (TI). (It would be helpful
    to know the basis for estimating the uncertainty of these
    calculations.) The results indicate that there is a fairly small,
    but systematic discrepancy between the two methods. In particular,
    in almost all cases, GC-AdResS underestimates the magnitude of the
    chemical potential. It would be very interesting and helpful if
    the authors could investigate and provide insight into the origin
    of this discrepancy.}}

The only reason that came to our mind is that such a systematic difference may come from the interaction cut off.
In fact, currently, we are using the reaction field method for electrostatics and thus the accuracy may indeed depend on that.

In the revised version we have added a plot that shows the convergence
of $\mu^{ex}$ as a function of the interaction cut-off.  Indeed we can
systematically improve our results if we go to higher cut-off (for
systems whose electrostatics is particularly relevant).  However, we
also show that for a cut-off of 1.4 nm (standard in full atom
simulations) we already reach a quite accurate value.  This is
reported and commented in the revised version.


{\color{blue}{\it I am somewhat confused by the notation for
    $F_{th}$. Is this a vector quantity? Is the integral on page 3
    giving a vector or a scalar? The notation seems confusing and
    inconsistent. Also, I am confused by the first equality sign after
    $F_{th}$. It would be helpful if the authors could clarify this.}
}

We apologize for the misprint, the term $F_{th}=$ has been removed, and now the quantity is well defined.


{\color{blue}{\it On page 5, the authors indicate that equipartition
    implies a contribution to the chemical potential of 1/2 kT for
    each atomic degree of freedom (dof). I would expect that this
    would be true if each atomic dof contributes quadratically to the
    Hamiltonian. It would be helpful if the authors could clarify why
    this would be so for dof's that do not contribute quadratically to
    the Hamiltonian.}}

We must again apologize, we reported the same written in Ref.5 where
we gave a general explanation in case molecules had also vibrational
degrees of freedom. Of course here we have rigid model and rotational
DOF's are not quadratic. The kinetic part related to rotations can
actually be written analytically, however we found out that it is not
very practical and thus the numerical trick of double resolution also
in the GC region simplifies the procedure.  For this reason we have
left only this part and removed the comment about the equipartition
theorem.

{\color{blue}{\it  It would be helpful if the authors could clarify why/how eq 3 generalizes to eq 4 in the case of a multicomponent system}}

We have added an analytic derivation of the extension to
multicomponent systems in the appendix.  We have tried to give the
essential physical arguments, but cannot avoid the lengthy discussion
otherwise crucial point may be missed.

 













\end{document}
