\documentclass[12pt]{article}
\usepackage{german}
\usepackage{a4}
\usepackage{graphicx}
\usepackage{pslatex}
\usepackage[]{color}
\usepackage{amsmath,amssymb,amsfonts,latexsym,graphicx}

\pagestyle{empty}
\parindent0cm
% \addtolength{\textheight}{2cm}
% \addtolength{\textwidth}{1cm}
\renewcommand{\sfdefault}{cmss}
\renewcommand{\familydefault}{\sfdefault}
\newcommand{\redc}[1]{{\color{red} #1}}            
\newcommand{\recheck}[1]{{\color{red} #1}}            

\newcommand{\vect}[1]{\textbf{\textit{#1}}}
\newcommand{\dof}{{\textrm{DOF}}}
\newcommand{\AT}{{\textrm{{AT}}}}
\newcommand{\EX}{{\textrm{EX}}}
\newcommand{\CG}{{\textrm{CG}}}
\newcommand{\HY}{{\Delta}}
\newcommand{\rdf}{{\textrm{rdf}}}
\newcommand{\thf}{{\textrm{th}}}
\newcommand{\res}{{\textrm{rep}}}
\newcommand{\ext}{{\textrm{extra}}}
\newcommand{\exc}{{\textrm{exc}}}
\newcommand{\thermo}{{\textrm{Q}}}
\newcommand{\hadress}{{\textrm{H}}}
\newcommand{\dadress}{{\textrm{D}}}

\newcommand{\kopf}{\noindent
\raisebox{2.5cm}[0cm][0cm]
{
\parbox[c]{0.88\textwidth}
{
\begin{center}
\textsc{\large Institute for Mathematics, Freie Universit\"{a}t Berlin}\\
\textsc{ PD Dr. Luigi Delle Site}
\end{center}
}
}}
%\raisebox{2cm}[0cm][0cm]{\rule{\textwidth}{0.2mm}}}
\begin{document}

\kopf
\vspace{2cm}
\hfill\raisebox{1.5cm}[0cm][0cm]
{\parbox[t]{0.34\textwidth}
{e-mail:\\
{\small luigi.dellesite@fu-berlin.de}\\
Arnimallee 6\\
D-14195 Berlin\\
Tel:\ +49 (0)30 838 75775 \\
Fax:\ + 49 (0)30 838 75412 \\
}}

\vspace{3cm}
\noindent\raisebox{1.5cm}[0cm][0cm]
{\fbox{\parbox[t]{0.6\textwidth}
{Editor of {\it The Journal of Chemical Physics}\\
}}}
\vspace{1cm}

\textbf{Resubmission of the paper: ``Chemical potential of liquids and mixtures via Adaptive Resolution Simulation''}

\vspace{1cm}
\parskip 2ex
Dear Editor\\
We would like to thank both referees for the criticisms and suggestions.
We agree on several points, however tend to strongly disagree on some crucial ones.\\
Below please find a reply to each of the concerns raised by the referees and the
corresponding modification of the manuscript (where required).\\
Moreover one section is dedicated to comments that are for you and are properly labeled.\\

{\bf For the Editor:}

Regarding the comment of {\bf Reviewer 1}:

{\it  It is not easy to judge the relevance of this
  work because it might contain enough simulations, but not new information. In fact, its main contribution is a table, with comparison with reported data.}

In the previous paper published about one year ago in Physical Review X,
we have developed what we have {\bf openly} proposed in that paper as a {\bf very rigorous approach} where all the formulas were derived analytically and checked mathematically.

The whole theory is based on two assumptions:

(1) Thermodynamic limit for the \redc{atomistic and the coarse-grained region. (i.e. the number of molecules in both of the two
  regions are infinitely large, and the coarse-grained region is infinitely larger than the atomistic region)}  

(2) Hybrid region considered as a thin filter compared to the size of the two other regions.

In a simulation of course these conditions cannot be met at the ideal
level, 
thus the idea of this paper was that of treating real systems
in order to explore the numerical {\bf accuracy} that the formulas can
provide. Thirteen systems were studied with both AdResS and with the
thermodynamic integration, this {\it per se} is a major numerical
effort, \recheck{I would delete ``even for large computer facilities'', see the reason below}, however, the systems were
carefully chosen as representative of different classes of liquids and
mixtures characterized by different physics. The {\bf new information}
consists of thirteen values of the chemical potential which are
reproduced by AdResS, thus showing that the method is trustable for
such kind of calculations and that the hypothesis (1) and (2) above
can be easily met in \recheck{systems of relatively small sizes (see Table~II of the manuscript). Actually
  all simulations reported were performed on a 8-core workstation.}
\recheck{Comment from HAN: the thermodynamic limit does not imply extremely large system,
  rather, comparatively small systems that can be simulated on desktop-workstations.
  This is actully the advantage of our method, and one of the important information
  delievered by the paper.}

Moreover, regarding the comment below:

{\it On the other hand, the paper lacks important pieces of
  information and many incomplete statements and definitions (see
  below). There is also a theoretical weakness in the sense that the
  main equation used to evaluate the chem. potential seems to be
  derived in an ad hoc fashion, and ``proven numerically'', as the
  authors wrote. It is not easy to rely on an equation/method which
  lacks a solid theoretical background (mostly in statistical
  physics)}

Some of the information asked from the referee is indeed missing,
i.e.~description of the coarse-grained model. \recheck{Actually  this is} a
further advantage of our approach: One does NOT need to derive the
\recheck{coarse-grained models}, but simply takes spherical models interacting via a WCA
potential. It is in fact the thermodynamic force which makes them
compatible with the corresponding full atomistic models. We will add this part in
the Appendix.

However, we tend to strongly disagree on the comment about the lack of
solid theoretical background. \recheck{One can refer to all details of
  the theoretical developement that has alreadly been published in
  Phys.Rev.X.,} and we would like very much to avoid to report again
the whole derivation \recheck{ in this paper, since it would be redundant.}

If, instead, the referee means that he found our paper on Phys.Rev.X
was not solid enough or wrong, then clear scientific arguments must be
brought forward and eventually a comment to the PRX paper should be
done. We did not find helpful the \recheck{words like ``{\bf seems to
    be derived}''}. In the PRX paper, we claimed that the equations
were \recheck{derived \textbf{rigorously} and {\bf exactly}},
\recheck{thus either the referee had found some mistakes in our
  previous paper (then this should be made clear), or he did not read
  the PRX paper.}

In conclusion, we will take several suggestions from {\bf Reviewer 1},
\recheck{however, we will reduce the additions about the
derivation of the formulas to the minimum. Mostly we refer to the exact
equations in the PRX paper}. \recheck{Indeed, we intend} that people focus on
the numerical calculations rather then on the analytical derivations
which can be found in full details in the PRX paper.

We are confident that regarding these points we have your understanding.

\vskip 2cm

Below we report the reply to each referee. The comment of the referees
are in italics.  In order to facilitate the work of the referee, the
additions/modifications in the text will be in red.


{\bf Reply to Reviewer 1:}

{\it ``..... The idea of calculating excess
  chemical potentials from adaptive resolution techniques is appealing}

We are pleased that the referee recognizes that the overall idea is interesting; we take this comment as a positive basis of our work.

{\it ``This work is in fact a sort of extension of Ref. 5, and its contribution is just to offer a check of the
statements, hypothesis and calculations made on that previous paper. It is not easy to judge the relevance of this
work because it might contain enough simulations, but not new information. In fact, its main contribution is a table, with comparison with reported data.}

In Ref.5, we claim to have developed a rigorous approach from the
point of view of physics, and gone even further by exploring how far
one could make it mathematically rigorous/exact. We have found that
under the hypothesis:

(1) Thermodynamic limit for the \redc{atomistic and the coarse-grained
  region. (i.e. the number of molecules in both of the two regions are
  infinitely large, and the coarse-grained region is infinitely larger
  than the atomistic region)}

(2) Hybrid region considered as a thin filter compared to the size of
the two other regions.

The method is mathematically rigorous and for some aspects even
exact. Conditions (1) and (2) are ideal, so it was important to know
how they can be met in \recheck{practical simulations, especially in
the mixture systems that the solute concentration is dilute.} The aim of this
paper is: {\bf to explore the numerical accuracy of the theory (for
  the chemical potential) proposed in Ref.5}.

We have done it for thirteen representative \recheck{I would remove ``and computationally expensive'', reason as before} systems.
We hope the referee can recognize that this effort is a non trivial one.


{\it On the other hand, the paper lacks important pieces of information and many incomplete statements and definitions (see below). There is also a theoretical
weakness in the sense that the main equation used to evaluate the chem. potential seems to be derived in an ad hoc fashion, and ``proven numerically'', as the authors wrote. It is not easy to rely on an equation/method which lacks a solid theoretical background (mostly in statistical physics)}

The sentence {\bf ``proven numerically''} has been used by the referee
as argument for an {\it ad hoc} formulation of the method, and our
impression is that he/she means that we constructed it based on
\recheck{what we see numerically}. \recheck{ In fact, it is not. In
  Ref.5, the theory was ({\bf explicitly}) derived from thermodynamic
  priciples, and was proved in a rigorous and exact way (in the due
  limits). The conclusion itself does not need further numerical verfication.
  The referee is legitimate to think that it is not rigorous, but
  then he/she should point out explicitly the weak points of Ref.5 and
  provide solid arguments.  However, in this case the referee should 
  correct/comment the paper of Ref.5. The numerical tests
  presented by this manuscript provide extra information on how our
  theory works in practial pure/mixutre systems, in which
  thermodynamic limit can only be approached, but never be reached. We believe this is a plus
  rather than a minus of our work. We are confident that the referee
  sees our point.
}

The sentence {\bf ``proven numerically''} has been used by the referee
as argument for an {\it ad hoc} formulation of the method, and our
impression is that he/she means that we constructed it in order to get
what we wanted. In fact it is not, the formula we have proven
numerically is actually an extra information. We do not need to use
it, in fact we have a technically rigorous way to check the heat given
by the thermostat; this formula just gives an interesting insight,
this is explicitly written in Ref.5. In this paper we have explored it
further and proved that works for other twelve systems. We believe
this is a plus rather than a minus of our work.

In general, as we have said above, we have presented our approach in
Ref.5 ({\bf explicitly}) as rigorous and exact (in the due limits).
The referee is legitimate to think that it is not, but then he/she
should point out explicitly the weak points of Ref.5 and provide solid
arguments.  However, in this case should be corrected/commented the
paper of Ref.5, the numerical result of this paper suggest that at the
simulation level the method cannot be that bad. We are confident that
the referee sees our point.

We found several comments extremely useful to improve the quality of the paper and performed several additional calculations in order to fulfill the request of the referee, we report below the description of all of them.

{\it (1) A description of the cg model is lacking. Please specify how
  do you treat and the interaction potential for cg molecules of
  solvent and solute. I suppose that both types of molecules (solvent
  and solute) free to cross the hybrid layer.  Please confirm. Please
  add a section about the cg model.}

We added it in the appendix. It must be noticed that we actually do
not need to derive a cg model. In fact our approach allows us to use
simple spherical models \recheck{(e.g.~interacting via the WCA potential)} whose size is about the vdW radius of the
specific molecule. It is
then the thermodynamic force that makes the thermodynamics of the cg
models compatible with the atomistic one. Molecules change their
resolution when they pass from one region to another.

{\it (2) Definitions of $V_{AT}$ and $V_{CG}$ are also lacking in terms of particle indexes. The model definition is important to distinguish between total energies and pair-wise terms, e.g. $V_{AT}(xij)$.}

\recheck{we can refer to the new appendix}

{\it (3) Figure 1. ``allows to write an exact Hamiltonian for the atomistic region and thus treat the system in a GC fashion.'' This phrase is
difficult to understand. The AT region is connected with the CG domain (via a transition layer which is not small, but much thicker than the
molecular size). Thus either one has a complete Hamiltonian for the whole AT+HYB+CG system, or else, there is no Hamiltonian. In other words, if
the interactions between AT and HYB, and CG molecules are not derived from H, then there is no Hamiltonian. Also, ``...treat the system in a GC
fashion'' is also difficult to understand (maybe the authors mean ``...consider the AT region as an open subsystem''). I would suggest
deleting this phrase and refer in the text to their theoretical findings in Ref. 5 about particle number distributions p(N),etc.}

Actually this is the very core of our derivation. If the referee think
that this is wrong, then Ref.5 is entirely wrong.

In fact we have extended the hybrid region adding a region equal to
the interaction cut off where the resolution is full atomistic, but
such a region is not considered part of the full atomistic region. As
a consequence all the molecules of the atomistic region will interact
with the hybrid region {\bf ONLY} at atomistic level (coarse-grained
molecules are in the region beyond the cut off one). In fact beyond
the cut off radius there is no interaction or better the interaction
is zero. As a consequence the molecules of the atomistic region have a
well defined Hamiltonian. This is the Hamiltonian of an open system,
\recheck{and is parametrized by the the coordinate of the molecules
  that are interacting with the atomistic region}. \recheck{It is comparable with  a
subsystem of a large full atomistic simulation}; the Hamiltonian of the
subsystem is well defined, without any artificial interpolation.  This
is clearly reported in Eq.12 of Ref.5.

Although we believe it is redundant we have reported the explicit form
of the Hamiltonian of the atomistic region also in this paper.

\recheck{{\bf HAN can you write it for me in Latex}}.
\recheck{{\bf See below}}
\begin{align}
  \mathcal H^{\AT}_{N_1} (\vect x_1; \vect x_2, N_2)
  =
  \sum_{i=1}^{N_1}\frac12 m_i\vect v_i^2
  + \sum_{i,j=1}^{N_1}\frac 12 V^\AT_{i,j}
  + \sum_{i=1}^{N_1}\sum_{j=N_1+1}^{N_1+N_2} V^\AT_{i,j},
\end{align}
where $V^\AT_{i,j}$ is the atomistic molecular potential, which is given by
\begin{align}
  V^\AT_{i,j} =
  \sum_{\alpha\in i}\sum_{\beta\in j} V^\AT(\vect r_\alpha - \vect r_\beta).
\end{align}
where index $\alpha$ and $\beta$ denotes the atoms on the
corresponding molecule.  $N_1$ and $N_2$ are the number of molecules
in the AT and atomistic part of the HY region, respectively. $\vect x_1$
and $\vect x_2$ are the phase space variables (joint of velocity and position)
in the  AT and atomistic part of the HY region, respectively.

{\it (4) Page 4. ``...reproduce the first order of the
  probability...'' and ``Higher orders...'' Could the authors specify
  what (small) quantity the order is referred to?, I think it could be
  orders in the density of solute molecules, or maybe in density
  variation with respect the average. Please be more clear when
  referring to ``orders'' elsewhere in the manuscript.}

\recheck{comment from HAN: why not add your short description on the
  orders to the paper? It is concise and would be convenient for the
  readers who are only interested in the numerical result, and do not
  have time to read Ref.5.}
  
This is clearly described in Ref.5 and we felt it would be redundant
to write it here once again; we have defined the first order of the
probability distribution of a system as its molecular density, a
function in three-dimension. The second order is the two-body
distribution (6-dimensional function) of which the radial distribution
function is a specific expression; the third order is the three-body
distribution (in 9-dimension) etc etc. The full probability is a
3N-dimensional function of which the different orders are a lower
dimensional approximation.

If the referee put this as a condition of rejection, then we would add
it, but we strongly believe that it is not needed for this paper.

{\it (5) Page 5. Auxiliary Hamiltonian. For the sake of clarity, the
  authors should write down this Hamiltonian they refer to, with full
  index details. Also, please write down the form of thermodynamic
  force in the auxiliary Hamiltonian. They mention that simulations
  using the auxiliary Hamiltonian were made with no thermostat. Did
  the authors check that the ``energy'' of this Hamiltonian is
  conserved? Please comment on this important issue.}

\recheck{Comment from HAN: Wo DO use a thermostat for the potential-interpolation
  simulation.  we should be honest on this point. And the prove in the
  appendix also assumes that ths thstem is thermostated. Tell me why you wrote ``
  we do not need to use a thermostat''? Or you simply mean that we in principle we
  do not need, but in practice, to generate the Boltzmann distribution we use it.}
  
  As before, it is reported in all details in Ref.5 and we thought it
  was redundant. The thermodynamic force must have exactly the same
  numerical formula of the standard AdResS, otherwise it would not
  work for our purposes: if two expressions are formally identical but
  one has the effect of the thermostat and the other not then the
  difference isolates the effect of the thermostat. For the
  Hamiltonian system we use the same formula and impose artificially
  the same density of the AdResS system thus we get a condition on the
  gradient of the molecular density in the transition region.
  
  Yes, we have checked the conservation of the energy. However this
  was preliminary to everything, in fact it must be clear that if the
  energy was not conserved, then we would also get completely wrong
  results (the heat would either be too low or too high and anyway
  would not converge).


  {\it (6) ``Keeps the density of particles across the system as in
    AdResS''. Do you mean at constant density?}

Yes


{\it (7) ``It must be noticed that the relation we obtained form the
  chem. pot. is similar to that obtained in... Ref[6]''. This
  statement is not clear. The equation this method proposes seems not
  similar to that of Ref. 6. In particular the term is quite different
  from which is the corresponding free energy difference appearing in
  Ref. 6. The authors should explicitly write out this essential term
  of the calculation, indicating summations with particle indexes. It
  would seem from the (not-written auxiliary Hamiltonian) that the
  particles index of the first ``w'' is not the same as that of the
  ``nabla w''. Please derive and clarify this term so anybody could
  understand it.

  The effect of discrepancies with Ref.6 are not simple to trace, but
  certainly not zero. The authors state: ``H-Adress and CG-Adress are
  the same up to first order'' (please again, in what?). A reference
  proving this statement is not mentioned, but I guess the authors
  refer to Ref. 7 for that prove. However after reading that paper
  (Ref. 7) I saw no prove of this (just essentially the same phrase,
  without prove). The authors should prove this statement here or
  delete this phrase, or comment about the lack of comparisons, at
  present.}

We see the point of the referee, perhaps we should do a paper with
explicit formulas and data to show similarity and differences between
the two methods.

However since this is not a relevant part of the paper, we decided to
remove any reference to similarities and differences with Ref.6.  We
will cite their second paper dealing with a mixture. We hope that the
referee is happy with the deletion of Ref.6 and addition of a new
reference.


{\it (8) Fig. 2. Please indicate the locations of the HYB domain. A
  difference of about 20 percent in density is not small, although
  *inside* the AT domain, the g(r) might not be too sensible to such
  density difference (as shown in the fig.). On the other hand,
  contrary with what the authors state below, a large transition
  region is beneficial, because molecules have larger space to adapt
  to the change in resolution. Please correct this.}

We disagree with the referee.
\recheck{The large deviation is due to the sampling error, because the
  number of TBA molecules in the system is very small (80, see the Table II).
  We could reach a much better accuracy of the density if we do longer simulations,
  but it would require much more computer time.}
% We could reach a much higher accuracy if
% we set the level of accuracy of the thermodynamic force higher (but
% requires more computer time)
However, we wanted to show that even with 20 percent of (maximum)
difference we get the chemical potential accurate and the density in
the AT region flat.
\recheck{comment from HAN: is the HY region goes
  from 0.5 nm to 3.2 nm, i.e.~the whole range of the figure 2? Then we
  do not show the density in the AT region is flat.}

Regarding the point
that a larger region is beneficial, we must disagree. This is based on
arguments (which we will give next) and long experience with such
systems. What is important in AdResS is that the AT and CG region are
large, in fact if they are large they are going to thermalize and in
general equilibrate substantially the hybrid region.  We have recently
made tests on related issues and found that if one uses a large
thermalized reservoir then small regions where the density or the the
temperature are not targeted, are automatically equilibrated by the
large reservoir. Moreover our analytic derivation suggests the same.
On the contrary a HYB region is charaterized by non physical
description of the potentials and thus it does not have as a reference
any thermodynamic state point. As a consequence the larger the HYB
region the larger the perturbation. Of course there is a minimal size
that is a region equal to the interaction cut off.  These aspects have
been explored and described in details in several previous papers
among which:

(1) Matej Praprotnik, Luigi Delle Site, and Kurt Kremer, .J.Chem.Phys. 123, 224106 (2005).

(2)Silvina Matysiak, Cecilia Clementi, Matej Praprotnik, Kurt Kremer and Luigi Delle Site, J.Chem.Phys. 128, 024503 (2008).

(3) S.Poblete, M.Praprotnik, K.Kremer and L.Delle Site,  J.Chem.Phys. 132, 114101 (2010).

{\it (9) Table I. What results of Ref. 15 corresponds to table I?,
  this is not mentioned. Results in Ref 12 and 15 were made at lower
  concentration, but the authors state that the excess chem pot is not
  sensible to the solute concentration in this regime. And say that
  ``we have verified that such consistency holds''. This is important
  though, and the authors should include a graph with the calculated
  excess chem. pot. against the solute concentration, showing that it
  goes to a plateau at low concentration consistent with the
  experimental value reported (at least for some of the cases
  considered). Error bars are crucial, but means nothing without the
  simulation times required to obtain them. The authors should include
  a more clear measure of this. A possible one could be showing the
  accumulated standard error against the simulation time and compare
  the IPM, the TI (free energy estimate using Bennet's method) and
  their own method. With the specification of the system size (table
  II) this already gives an idea of the computational costs.}

We do understand the concerns of the referee about the values referred
to paper in Ref.15, however all the values in the table for TI were
obtained by us.  We have checked (the sentence {\it ``we have verified
  such a consistency''}) that our values are close to those at lower
concentration.  We have also made the (expensive) calculation of the
concentration dependence and show that indeed goes towards a plateau
added the error etc etc.  The figure and related comments have been
added in the revised version of the manuscript.

However, at the current stage the comparison about the computational time required by AdResS and TI would not be fair.
In fact for this reason we have {\bf honestly} reported current numerical limitations in Section IV.

This paper is about the capability and numerical accuracy of the approach.

{\it (10) Sec. IV. ...a critical appraisal. This comment is connected with 8). In this section the authors state that their method is not as fast as TI if dilute systems are considered. I would just like to comment that this observation seems to conflict with the comment made in Ref. 7 (``orders of magnitude faster than IPM''). Indeed, Widom method is not very good for large molecules in liquids, this is well known. So, as far I understand, the author's conclusion is that the best option is to use TI for dilute systems while adaptive resolution pays off when dealing with many solute molecules. This observation is important and should be shown in a graph (e.g. standard error vs. simulation time for TI and CG-Adress, in the dilute case and less diluted case)}

NO, we only say that in the future, AdResS, once its implementation is optimized, may even offer a better way to calculate the chemical potential in some situations, at the moment, although in very recent work we have reached a speed up factor of more than two w.r.t. full atomistic simulation, the computational implementation of AdResS in Gromacs is still not optimal. What we instead claim (for sure) is that if you do in any case an AdResS simulation, you automatically get the chemical potential, that is this is an automatic feature of the simulation method. This is reported clearly in the paper and we believe that it does not need further clarifications.\\

{\it (11) Recent works: an extension to the Hamiltonian based method the authors refer to, appeared in ``Phys. Rev. Lett. 2013, 111, 060601''. This work should be cited because it precisely deal with mixtures being in close relation with this work.}

OK, as said before we cite this instead of Ref.6.



{\bf Reply to Reviewer 2}:\\

 {\it I would recommend that the authors consider the following minor revisions. Most significantly, it would be very helpful for the authors to discuss the apparently systematic errors that arise in their estimation of the chemical potentials. The following provides further details and minor comments.

Table 1 compares the chemical potentials determined from GC-AdResS and thermodynamic integration (TI). (It would be helpful to know the basis for estimating the uncertainty of these calculations.) The results indicate that there is a fairly small, but systematic discrepancy between the two methods. In particular, in almost all cases, GC-AdResS underestimates the magnitude of the chemical potential. It would be very interesting and helpful if the authors could investigate and provide insight into the origin of this discrepancy.}

The only reason that came to our mind is that such a systematic difference may come from the interaction cut off.
In fact, currently, we are using the reaction field method for electrostatics and thus the accuracy may indeed depend on that.\\
We found this point very interesting and studied the dependence of our results from the cut off. Indeed it seems that, etc etc...... {\bf WHEN ANIMESH IS READY}\\


{\it  I am somewhat confused by the notation for $F_{th}$. Is this a vector quantity? Is the integral on page 3 giving a vector or a scalar? The notation seems confusing and inconsistent. Also, I am confused by the first equality sign after $F_{th}$. It would be helpful if the authors could clarify this.}

We apologize for the misprint, the term $F_{th}=$ has been removed\\


{\it On page 5, the authors indicate that equipartition implies a
  contribution to the chemical potential of 1/2 kT for each atomic
  degree of freedom (dof). I would expect that this would be true if
  each atomic dof contributes quadratically to the Hamiltonian. It
  would be helpful if the authors could clarify why this would be so
  for dof's that do not contribute quadratically to the Hamiltonian.}

We must again apologize, we reported the same written in Ref.5 where
we gave a general explanation in case molecules had also vibrational
degrees of freedom. Of course here we have rigid model and rotational
DOF's are not quadratic. The kinetic part related to rotations can
actually be written analytically, however we found out that it is not
very practical and thus the numerical trick of double resolution also
in the GC region simplifies the procedure.  For this reason we have
left only this part and removed the comment about the equipartition
theorem.

{\it  It would be helpful if the authors could clarify why/how eq 3 generalizes to eq 4 in the case of a multicomponent system}

\recheck{\bf Added below... I do not really know what to write for him/her.}

 We denote the volume of the system by $V$ and the temperature by
$T$. It is divided into atomistic~($\AT$), hybrid~($\HY$), and
coarse-grained~($\CG$) region, the volume of which are denoted by
$V_1$, $V_2$ and $V_3$, respectively.  We assume two different types
of molecules are mixed in the system, and the number of molecules are
$N^1$ and $N^2$, respectively. The superscript ``1'' and ``2'' in the
paper denotes the two component.  We further denote the number of type
1 molecules in the atomistic, hybrid, and coarse-grained regions by
$N^1_1$, $N^1_2$ and $N^1_3$, respectively. While for type 2 molecules
the notations are $N^2_1$, $N^2_2$ and $N^2_3$, respectively.  By
assuming the hybrid region is infinitely small comparing with the
atomistic and coarse-grained regions, we have the following
constrains:
\begin{align}
  V &= V_1 + V_3\\
  N &= N^1 + N^2\\
  N^1 &= N^1_1 + N^1_3\\
  N^2 &= N^2_1 + N^2_3
\end{align}
By calculating the thermodynamic forces for both of the components, which
are denoted by $\vect F^1_\thf$ and $\vect F^2_\thf$,
we impose the correct density profile to the system:
\begin{align}
  \rho_\HY^1 &= \rho_\AT^1 = \rho_\CG^1\\
  \rho_\HY^2 &= \rho_\AT^2 = \rho_\CG^2
\end{align}
We further denotes the work done by the hybrid region on the two types
of molecules by $\omega_0^1$ and $\omega_0^2$, respectively.
To find out the chemical potential difference between AT and CG resolutions,
the same idea as presented in Sec.III.C of Ref.5 can be extended
to the two component system. This extension is straightforward and trivial. We do not want
to write them down, because one can just step-by-step follow the
deriviation for the one component system, and no new technique is needed.
In the end, we proved
\begin{align}
  \mu^1_\AT(N^1_1, N^2_1, V_1, T) &= \mu^1_\CG(N^1_1, N^2_1, V_1, T) - \omega^1_0\\
  \mu^2_\AT(N^1_1, N^2_1, V_1, T) &= \mu^2_\CG(N^1_1, N^2_1, V_1, T) - \omega^2_0
\end{align}
Now, $N^1_1$ and $N^2_1$ denotes the averaged number of type 1 and
type 2 molecules in the atomistic region.  Under the thermodynamic
limit, this number maximize the Helmholtz free energy and is the one
of statistical importance, see more discussions in
Ref.5. We do not distinguish these notations from
before, because the meaning is clear from the context.











Sincerely\\

Luigi Delle Site





\end{document}
