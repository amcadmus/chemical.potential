%%%%%%%%%%%%%%%%%%%%%%% file template.tex %%%%%%%%%%%%%%%%%%%%%%%%%
%
% This is a template file for The European Physical Journal Special Topics
%
% Copy it to a new file with a new name and use it as the basis
% for your article
%
%%%%%%%%%%%%%%%%%%%%%%%% Springer-Verlag %%%%%%%%%%%%%%%%%%%%%%%%%%
%
\documentclass[epjST]{svjour}
%
\usepackage{graphics}
\usepackage{color, colortbl}
\usepackage{amsmath,amssymb}
\usepackage{graphicx}
\usepackage{tabularx}

\newcommand{\recheck}[1]{{\color{red} #1}}
\newcommand{\redc}[1]{{\color{red} #1}}
\newcommand{\bluec}[1]{{\color{red} #1}}
\newcommand{\greenc}[1]{{\color{green} #1}}
\newcommand{\vect}[1]{\textbf{\textit{#1}}}
\newcommand{\dd}[1]{\textsf{#1}}
\newcommand{\fwd}[0]{\textrm{fw}}
\newcommand{\bwd}[0]{\textrm{bw}}
\newcommand{\period}[0]{T_{\textrm{P}}}
\newcommand{\ml}[0]{\mathcal {L}}
\newcommand{\mo}[0]{\mathcal {O}}
\newcommand{\mbp}[0]{\mathbb {P}}
\newcommand{\mh}[0]{\mathcal {H}}
\newcommand{\dist}[0]{\textrm {dist}}
\newcommand{\AT}[0]{\textrm{AT}}
\newcommand{\HY}[0]{\textrm{HY}}
\newcommand{\CG}[0]{\textrm{CG}}
\newcommand{\EX}[0]{\textrm{EX}}
\newcommand{\moleidxone}[0]{i}
\newcommand{\moleidxtwo}[0]{j}
\newcommand{\atomidxone}[0]{\alpha}
\newcommand{\atomidxtwo}[0]{\beta}
\newcommand{\thf}{{\textrm{th}}}
\newcommand{\rdf}{{\textrm{rdf}}}
\newcommand{\rep}{{\textrm{rep}}}
\newcommand{\dof}{{\textrm{DOF}}}
\newcommand{\exc}{{\textrm{extra}}}
\newcommand{\inv}{{\textrm{equi}}}

% \usepackage{amsfonts}
% \newcommand{\tickYes}{\checkmark}
% \usepackage{pifont}
% \newcommand{\tickNo}{\hspace{1pt}\ding{55}}
% \definecolor{MyGray}{gray}{0.85}

%
\begin{document}
%
\title{Adaptive Resolution Simulation: Theoretical Fundations and Open Questions}
\author{Han Wang\inst{1}\fnmsep\thanks{\email{han.wang@fu-berlin.de}}}
%
\institute{Zuse Institut Berlin}
%
\abstract{
Insert your abstract here.
} %end of abstract
%
\maketitle
%


\section{The two ways  of designing the hybrid region}

The AdResS uses a weighting function $w(\vect r)$ that is a scalor
function defined over the simulation region to denote the resolution
of the system. In the atomistic region the value of the
weighting function is usually set to 1, while in the coarse-grained
region it is usually set to 0. In between, there is a bybrid region
that is denoted by $\Omega_\HY$
(or called transition region), where the molecules
has both the atomistic and coarse-grained resolutions, and the weighting function
goes smoothly from 1 to 0. The way of designing the weighting function
is not unique, one possible and perhaps the most popular choice is
\begin{equation}\label{eqn:old-w}
  w(\vect r) =
  \left\{
    \begin{alignedat}{3}
      &1 &\quad& \vect r \in \Omega_\AT\\
      &\cos^2\Big[\frac{\pi}{2 } \cdot \frac{\dist(\vect r, \Omega_\AT)}{d_\HY}\,\Big] && \vect r \in\Omega_\HY \\
      &0 &    & \vect r \in \Omega_\CG 
    \end{alignedat}
  \right.
\end{equation}
where $\dist(\vect r, \Omega_\AT)$ is the distance between the
position $\vect r$ and the atomistic region, which is defined
by $\dist(\vect r, \Omega_\AT) = \min_{\vect s\in\Omega_\AT} \vert
\vect r - \vect s \vert$.  $d_\HY$ is the thickness of the hybrid
region. We always assume the thickness of the hybrid region is
uniform, which indicates that the weighting function smoothly vanishes
at the boundary between the hybrid and the coarse-grained regions.
Since all interactions (including the electrostatic interction) are treated
by cut-off method, the weighting function requires the thickness of
the hybrid region being at least one cut-off radius (denoted by $r_c$). 
The benefite of this setting is that the atomistic
region is not directly interacting with the coarse-grained region,
so there is no extra work for modeling the interactions between the atomistic
and coarse-grained resolutions.

Both the atomistic and coarse-grained intermolecular interactions are
well defined in the atomistic region and coarse-grained regions,
respectively.  To setup an AdResS simulation, one needs to define the
interactions in the hybrid region. There are in general two
possibilities: force interpolation and the potential interpolation.
The force interpolation approach defines the hybrid force between two
molecules (saying $\moleidxone$ and $\moleidxtwo$) by a linear interpolation
between the atomistic (denoted by ${\vect F}_{\moleidxone\moleidxtwo}^{\AT}$) and coarse-grained (denoted by ${\vect F}_{\moleidxone\moleidxtwo}^{\CG}$) forces.
\begin{align}\label{eqn:f-f-interpol}
  {\vect F}_{\moleidxone \moleidxtwo}=w_\moleidxone w_\moleidxtwo{\vect F}_{\moleidxone\moleidxtwo}^{\AT}+(1-w_\moleidxone w_\moleidxtwo){\vect F}^{\CG}_{\moleidxone\moleidxtwo} 
\end{align}
where $w_\moleidxone = w(\vect r_\moleidxone)$ is the weighting function
measured at the center-of-mass (COM) position of molecule $\moleidxone$.
It should be noted that the AdResS  force interpolation is not conservative, i.e.~there does not
exist a potential so that the force interpolation is derived by taking the negative gradient of the potential~\cite{praprotnik2011comment,dellesite2007some}.
It is easy to show that the force interpolation satisfies the momentum conservation: Since
both the atomistic and coarse-grained forces are subject to the Newton's third law, i.e.~${\vect F}_{\moleidxone\moleidxtwo}^{\AT} = - {\vect F}_{\moleidxtwo\moleidxone}^{\AT}$
and ${\vect F}_{\moleidxone\moleidxtwo}^{\CG} = - {\vect F}_{\moleidxtwo\moleidxone}^{\CG}$, from the definition Eq.~\eqref{eqn:f-f-interpol} we have ${\vect F}_{\moleidxone\moleidxtwo} = - {\vect F}_{\moleidxtwo\moleidxone}$.

The potential interpolation approach defines the hybrid energy by the same idea:
\begin{align}\label{eqn:v-v-interpol}
  {V}_{\moleidxone \moleidxtwo}=w_\moleidxone w_\moleidxtwo{V}_{\moleidxone\moleidxtwo}^{\AT}+(1-w_\moleidxone w_\moleidxtwo){V}^{\CG}_{\moleidxone\moleidxtwo} 
\end{align}
The intermolecular force is therefore calculated by taking the negative gradient on the potential ${\vect F}^V_{\moleidxone \moleidxtwo}= -\nabla_{\moleidxone}{V}_{\moleidxone \moleidxtwo}$,
which is explicitly written as
\begin{align}\label{eqn:f-v-interpol}
  {\vect F}^V_{\moleidxone \moleidxtwo}
  &=w_\moleidxone w_\moleidxtwo{\vect F}_{\moleidxone\moleidxtwo}^{\AT}+(1-w_\moleidxone w_\moleidxtwo){\vect F}^{\CG}_{\moleidxone\moleidxtwo}  - \nabla w_\moleidxone\cdot w_\moleidxtwo (V^\AT_{\moleidxone \moleidxtwo} - V^\CG_{\moleidxone \moleidxtwo})
\end{align}
Where $\nabla w_\moleidxone$ is the gradient of the weighting function measured at position $\vect r_\moleidxone$,
and the superscipt ``$V$'' denotes that the force is defined
by the potential interpolation.
By  definition~\eqref{eqn:f-v-interpol}, the force of potential interpolation is conservative. 
The difference between the force and potential
interpolations lies in the last term  of Eq.~\eqref{eqn:f-v-interpol}, which is a force
acting along the direction of decreasing weighting function.
It should be noted that
the last term breaks the Newton's third law, therefore the force of potential interpolation
does not conserves the momentum.

\begin{table}
  \centering
  \caption{A summary of the fundamental conservation laws in mechanics 
    satisfied by the normal molecular system and AdResS systems defined by force and potential interpolations.}
  \label{tab:conv-laws}
  \begin{tabular*}{0.8\textwidth}{@{\extracolsep{\fill}}lcc}\hline\hline
    &         Momentum Consv.     &       Energy Consv. \\
    Normal molecular System     &       {Yes}        &       {Yes}\\      
    AdResS force interpol.   &       {Yes}        &       {No}\\      
    AdResS potential interpol.  &     {No}  &       {Yes}\\\hline\hline
  \end{tabular*}
\end{table}
The fundamental conservation laws satisfied by both the force and potential interpolations
are summarized in Tab.~\ref{tab:conv-laws}. Unlike normal molecular systems, in which both
the momentum and energy are conserved, neither of the AdResS systems conserves
both laws. The breaks of these mechanics conservations have substantial influence on the thermodynamic
properties of the AdResS systems (see the discussions in Sec.~\ref{sec:thermodynamic}). The choice of the interpolation scheme depends on
the practical requirement, for example, in some systems it is crucial to have the momentum
conservation, then the force interpolation is prefered, and vice versa.
In equilibrium case, we will show later in Sec.~\ref{sec:statistical}
that both approaches approximately sample the grand-canonial ensamble.

When the intermolecular force is defined by either
Eq.~\eqref{eqn:f-f-interpol} or \eqref{eqn:f-v-interpol}, the total
force exserts on one molecule is given by the normal adding up rule:
\begin{align}
  \vect F_\moleidxone = \sum_\moleidxtwo \vect F_{\moleidxone\moleidxtwo}, \quad \vect F^V_\moleidxone = \sum_\moleidxtwo \vect F^V_{\moleidxone\moleidxtwo}
\end{align}
On the force on the atomistic DOFs of a hybrid molecules is distributed 
by, taking atom $\atomidxone$ on molecule $\moleidxone$ for example,
\begin{align}
  \vect F_\atomidxone = \frac{m_\atomidxone}{M_\moleidxone}\vect F_\moleidxone, \quad   \vect F^V_\atomidxone = \frac{m_\atomidxone}{M_\moleidxone}\vect F^V_\moleidxone 
\end{align}
where $m_\atomidxone$ is the mass of atom, and $M_\moleidxone$ is the total mass of molecule $\moleidxone$, i.e.~$M_\moleidxone = \sum_{\atomidxone\in\moleidxone}m_\atomidxone$.

Before discussing the statistical and thermodynamic properties of AdResS, a few comments on the
choice of the weighting function should be added. 
In Ref.~\cite{wang2012adaptive} the authors introduced a modified weighing function that
introduce a buffer region so that the molecules with the atomistic resolution
interacts with the hybrid region via the atomistic intermolecular interaction:
\begin{equation}\label{eqn:new-w}
  w(\vect r) =
  \left\{
    \begin{alignedat}{3}
      &1 &\quad& \vect r \in \Omega_\AT\\
      &1 && \vect r \in\Omega_\HY, \dist(\vect r, \Omega_\AT) < r_c \\
      &\cos^2\Big[\frac{\pi}{2 } \cdot \frac{\dist(\vect r, \Omega_\AT) - r_c}{d_{{\HY}} - r_c}\,\Big] && \vect r \in\Omega_\HY, \dist(\vect r, \Omega_\AT) \geq r_c\\
      &0 &    & \vect r \in \Omega_\CG 
    \end{alignedat}
  \right.
\end{equation}
The minimum thickness of the hybrid region is therefore $2r_c$.
It worth noting that the coarse-grained region does not requires a buffer in the hybrid region, because
by definition the coarse-grained region interacts with the hybrid region via the coarse-grained intermolecular interaction.
The new definition~\eqref{eqn:new-w} is crucial for the equilibrium statistical properties of the AdResS,
however, the extra cost of is spent in the buffer $\{\vect r \vert \vect r \in\Omega_\HY, \dist(\vect r, \Omega_\AT) < r_c\}$,
which is treated in atomistic resolution. This cost is relatively small for systems with large atomistic region.

\section{Equilibrium statistical properties of adaptive resolution simulation}
\label{sec:statistical}

To consider the accuracy of the AdResS simulation, we always compare it with a
atomistic reference system, which is of the same size as the AdResS system, and contains the same number of molecules.
If the statistal properties of the atomistic region in the AdResS simulation
is same as the corresponding subregion in the atomistic reference, then
the atomistic region is embeded into the AdResS system as if it were embeded in to
an atomistic environment, and the AdResS simulation is of good accuracy.

We denote the number of molecules, volume and temperture of the system by $N$, $V$ and $T$, respectively.
The atomistic reference system has exactly the same set of variables, and its equilibrium state  is
the desired equilibrium.
% The word ``desired'' means that
% the atomistic region of the AdResS system should reproduce the equilibrium properties corresponding subregion of the reference
% system.
The thermodynamic variables for subregions are specified by the subscript,
for example, the those of the atomistic region are 
$N_\AT$, $V_\AT$ and $T_\AT$. Those of the hybrid and coarse-grained regions
are denoted by adding ``$\HY$'' and ``$\CG$'', respectively.
Identities $N = N_\AT + N_\HY + N_\CG$ and $V = V_\AT + V_\HY + V_\CG$ obviously hold
in the AdResS system.
In equilibrium, the temperature is uniform across the system: $T = T_\AT = T_\HY = T_\CG$.
This is achieved in practice by coupling the whole system to a Langevin thermostat.
The pressure and chemical potential of the atomistic and coarse-grained regions
are denoted by $\{p_\AT, \mu_\AT\}$
and $\{p_\CG, \mu_\CG\}$, respectively.
The DOFs of molecules indexed $\moleidxone$ are $\vect x_\moleidxone = \{\vect r_\moleidxone, \vect p_\moleidxone\}$, where
$\vect r_\moleidxone$ denotes the generalized coordinates and $\vect p_\moleidxone$ denotes the corresponding momenta.
All DOFs of the system is denoted by $\vect x = \{\vect x_1, \cdots, \vect x_N\}$. Without lost of
generality, the first molecules index by $\{1, \cdots, N_\AT\}$ are in the atomistic region, then $\{N_\AT+1, \cdots, N_\AT + N_\HY\}$ in hybrid region, and
the last $N_\CG$ molecules $\{N-N_\CG+1, \cdots, N\}$ are in the coarse-grained region. The corresponding DOFs are denoted by $\vect x_\AT$, $\vect x_\HY$ and $\vect x_\CG$.
\redc{The atomistic, hybrid and coarse-grained have different DOFs, a comment needed!}

\begin{figure}
  \centering
  \includegraphics[width=0.3\textwidth]{figs/system.thermo/system.eps}
  \caption{A schematic plot of the AdResS system in thermodynamic
    equilibrium. The number of molecules, volume and temperature of
    the atomistic and coarse-grained regions are denoted by $\{N_\AT,
    V_\AT, T\}$ and $\{N_\AT, V_\AT, T\}$, respectively. The filter
    allows free exchange of molecules between atomistic and
    coarse-grained regions.}
  \label{fig:system-thermo}
\end{figure}

We consider the system in the thermodynamic limit: both the atomistic
and the coarse-grained regions are infinitely large, and at the same
time, the atomistic region is much larger than the coarse-grained
region. The hybrid region is much smaller than both the atomistic and
coarse-grained regions.  Therefore, the number of molecules in the
three regions satisfies $N_\CG\gg N_\AT\gg N_\HY$.  The coarse-grained
region can be treated as an infinitely large particle and energy
reservior of the atomistic region, and the hybrid region is an
infinitely thin filter that changes molecular resolution when a
molecule passes by (see Fig.~\ref{fig:system-thermo}). In the
thermodynamic limit, if there were no change of resolution (or
considering the full atomistic reference system), it is obvious that
the subregion corresponds to the atomistic region samples the
grand-canonial ensemble.  The question regarding the accuracy of
AdResS can be asked by: How accurately the atomistic region samples
the grand-canonial ensemble in the thermodynamic limit. Being more
specifically, we want to prove the probability density of the atomistic region satisfies
\begin{align}\label{eqn:dist-0}
  p(\vect x_\AT, N_\AT) \approx \frac{1}{\mathcal Z} \exp\Big\{{\beta\mu^\ast_\AT N_\AT - \beta \mh^\AT(\vect x_\AT)} \Big\}
\end{align}
where $\mu^\ast_\AT$ is the chemical potential of reference system in
desired equilibrium, $\mathcal Z$ is the partition function
normalizing the probability density, and $\mh^\AT$ is the atomistic
Hamiltonian defined in the the subregion of the reference system that
corresponds to the atomistic region of the AdResS system.

In stead of directly proving Eq.~\eqref{eqn:dist-0}, Ref.~\cite{wang2013grand}
suggest investigating the following equivalent equations:
\begin{align}\label{eqn:dist-1-0}
  p(\vect x_\AT \vert N_\AT) &\approx \frac{1}{Z_{N_\AT}} \exp\Big\{{- \beta \mh^\AT(\vect x_\AT)}\Big\}  \\\label{eqn:dist-1-1}
  p(N_\AT) & \approx \frac{ Z_{N_\AT}}{\mathcal Z} \exp\Big\{{\beta\mu^\ast_\AT N_\AT}\Big\}
\end{align}
where $Z_{N_\AT}$ is the canonical partition function for an atomistic
system with $N_\AT$ molecules
\begin{align}
  Z_{N_\AT} = \int d\vect x_{\AT} \exp\Big\{{- \beta \mh^\AT(\vect x_\AT)}\Big\}.
\end{align}
The identity of the conditional probability holds: $ p(\vect x_\AT, N_\AT)  = p(\vect x_\AT \vert N_\AT) \, p(N_\AT) $.

\subsection{The accuracy of the configurational probability density}
The configurational probability density~\ref{eqn:dist-1-0} is further splitted as
\begin{align}
  p(\vect x_\AT \vert N_\AT) =
  \sum_{N_\HY} \int {d}\vect x_\HY\,
  p(\vect x_\AT \vert N_\AT; \vect x_\HY, N_\HY)\cdot
  p(\vect x_\HY, N_\HY\vert N_\AT)  
\end{align}
The first probability density in the integral is the probability density
of atomistic DOFs conditioned on the number of atomistic molecules and all
hybrid DOFs.
It can shown that if
(1) all interactions in the system are cut-offed;
(2) The atomistic region interacts with the bybrid region only in an atomistic way;
(3) The system is short-range correlated, and the correlation between the atomistic
and the coarse-grained regions is negligible, then the probability density $p(\vect x_\AT \vert N_\AT; \vect x_\HY, N_\HY)$
is approximated by
\begin{align}\label{eqn:dist-conf-0}
  p(\vect x_\AT \vert N_\AT; \vect x_\HY, N_\HY)
  \propto
  \exp\Big\{-\beta \mh^\AT(\vect x_\AT; \vect x_\HY, N_\HY) \Big\},
\end{align}
where 
\begin{align}
  \mh^\AT(\vect x_\AT; \vect x_\HY, N_\HY)
  =
  \sum_{i=1}^{N_\AT}\frac12 m_i\vect v_i^2 +
  \sum_{i,j=1}^{N_\AT} \frac12 V^\AT(\vect r_{ij}) +
  \sum_{i=1}^{N_\AT}\sum_{j=N_\AT+1}^{N_\AT + N_\HY} V^\AT(\vect r_{ij})
\end{align}
is the atomistic Hamiltonian with parameters $\vect x_\HY, N_\HY$. It
should be noted that the probability density~\eqref{eqn:dist-conf-0}
is identical to that of the reference system with a subregion that
corresponds to the hybrid region.
The probability density and local Hamiltonian
can be written down for the coarse-grained region analogically.

In general, the probability density $p(\vect x_\HY, N_\HY\vert N_\AT)$
is not the same as the reference system.  For example, it has been
shown that the density distribution in the hybrid region deviates from
the desired one, even if the coarse-grained side is modelled to
reporduce the atomistic pressure~\cite{poblete2010coupling}.  However,
it is possible to raise necessary conditions that systematically
improves the accuracy of the probability density. The first necessary
condition is that the density profile of the hybrid region should be a
constant that is identical to the density of the atomistic reference
at desired equilibrium:
\begin{align}\label{eqn:necessary-1st}
  \rho_\HY(\vect r) = \rho_\AT^\ast = \frac NV
\end{align}
The second necessary condition is that the two body probability
density in the hybrid region is the same as the full atomistic
reference, In homogeneous and isotropic system, it is equivalent to
ask that the hybrid radial distribution function (RDF) should be the
same as that of the reference system:
\begin{align}\label{eqn:necessary-2nd}
  g_\HY(r) = g^\ast_\AT(r).
\end{align}
It is fully justified to systematically raise the necessary conditions
upto $m$-th multibody probability density, for example the third necessary condition would be
\begin{align}
  C^{(3)}_\HY = C^{(3)\ast}_\AT
\end{align}
where $C^{(3)}$ is the three-body correlation.
Since the system is homogeneous
and isotropic, the three-body probability density is equivalent to
three-body correlation function. 
% However, it is not convenient in
% numerics, because the correction to the $m$-th multibody probability density
% needs a $m$-body correction force in general, which is computationally
% too expensive.

In practice, the density in the hybrid region is correct by the
thermodynamic force~\cite{fritsch2012adaptive}, which is applied on top of the AdResS intermolecular interactions:
\begin{align}
  \vect F_\moleidxone &= \sum_\moleidxtwo \vect F_{\moleidxone\moleidxtwo}  + \vect F_\moleidxone^\thf,  \\
  \vect F^V_\moleidxone& = \sum_\moleidxtwo \vect F^V_{\moleidxone\moleidxtwo} + \vect F_\moleidxone^{\thf,V},
\end{align}
where the thermodynamic forces is a one-body force defined over space,
i.e.~$\vect F^\thf_i = \vect F^\thf(\vect r_\moleidxone)$ and $\vect F^{\thf, V}_\moleidxone = \vect F^{\thf,V}(\vect r_\moleidxone)$,
for force and potential interpolations, respectively.
The thermodynamic force is applied only in the hybrid region, and is
calculated by the following iterative scheme:
\begin{align}
  \vect F_{k+1}^{\thf(, V)} (\vect r) = \vect F_k^{\thf(, V)} (\vect r)-
  \frac{M}{\kappa(\rho^\ast_\AT)^2} \nabla\rho_k(\vect r)
\end{align}
where $k$ denotes the step of iteration. $\kappa$ is the
isothermo-compressibility. In equilibrium the density profile $\rho(\vect r)$ in the
atomistic and coarse-grained regions are constants, so the thermodynamic force
is updated only in the hybrid region. When the hybrid density is flat, the thermodynamic
force converges. An important subsequence of a flat hybrid density is that
\begin{align}
  \rho_\AT(\vect r) = \rho_\HY(\vect r) = \rho_\CG(\vect r) = \rho_\AT^\ast,
\end{align}
because the density profiles match at the atomistic-hybrid and
hybrid-coarse-grained boundaries.

In a numerical example of SPC/E water~\cite{berendsen1987missing}
system, it has been shown in Ref.~\cite{wang2013grand} that when the
thermodynamic force is applied, the AdResS atomistic region reporduces
mulit-body configurational probability density of the reference system
upto three-body correlation.
In principle, higher order of mulit-body correlation can be investigated in
a similar way. A surprising numerical observation is that although
only the thermodynamic force is applied, the second order necessary condition~\eqref{eqn:necessary-2nd}
is automatically fulfilled, because the RDF is identical to the reference system in the buffer hybrid region
$\{\vect r \,:\, \vect r \in\Omega_\HY, \dist(\vect r, \Omega_\AT) < r_c\}$ that is interacting with
the atomistic region. While the third necessary condition is not satisfied because deviation
in the three-body correlation presents in the buffer bybrid region.
It should be noted that there is no theoretical background why the second necessary
condition is satisfied by only using the thermodynamic force,
so one cannot expect the same benefit in other systems.

The second necessary condition~\eqref{eqn:necessary-2nd} is fulfilled by the
RDF correction $\vect F^\rdf$~\cite{wang2012adaptive}, which is a conservative two-body force added to the force interpolation~\eqref{eqn:f-f-interpol} as an extra term
that applies only in the hybrid region:
\begin{align}\label{eqn:rdf-corr-f}
  {\vect F}_{\moleidxone \moleidxtwo}
  =
  w_\moleidxone w_\moleidxtwo{\vect F}_{\moleidxone\moleidxtwo}^{\AT}
  +
  (1-w_\moleidxone w_\moleidxtwo){\vect F}^{\CG}_{\moleidxone\moleidxtwo}
  +
  w_\moleidxone w_\moleidxtwo(1-w_\moleidxone w_\moleidxtwo){\vect F}^{\rdf}_{\moleidxone\moleidxtwo}.
\end{align}
The RDF correction force is constructed from a RDF correction potential $V^\rdf$ by
\begin{align}
  \vect F_{\moleidxone\moleidxtwo}^\rdf = \vect F^\rdf(\vect r_{\moleidxone\moleidxtwo})
  = -\nabla_{\vect r} V^\rdf(r_{\moleidxone\moleidxtwo}),
\end{align}
which is calculated by the iterative Boltzmann inversion:
\begin{align}
  V_{k+1}^\rdf(r) = V_k^\rdf(r) + k_BT \ln \Big[\frac{g_k(r)}{g^\ast_\AT(r)}\Big] 
\end{align}
where $k$ is the step of iteration. When the RDF of $k$th iteration is
identical to that of the full atomistic reference $g^\ast_\AT(r)$, the
iteration converges. It has been shown that the iterative scheme of
thermodynamic force coupled with the iterative Boltzmann inversion RDF
is effective in correcting the density profile $g_\HY(\vect r)$ and
RDF $g_\HY(r)$ simultaneously~\cite{wang2012adaptive}.
For SPC/E water system when the RDF correction is applied, the third
necessary condition is automatically satisfied~\cite{wang2013grand}. Again, there is no
theoretical prove behind this phenomenum, and one cannot expect the same
benefit for other systems.

\noindent\textbf{Remark:} The RDF correction is developed only for the force interpolation scheme,
and a natural extension to the potential interpolation seems to be
\begin{align}
  {V}_{\moleidxone \moleidxtwo}
  =
  w_\moleidxone w_\moleidxtwo{V}_{\moleidxone\moleidxtwo}^{\AT}
  +
  (1-w_\moleidxone w_\moleidxtwo){V}^{\CG}_{\moleidxone\moleidxtwo}
  +
  w_\moleidxone w_\moleidxtwo(1-w_\moleidxone w_\moleidxtwo){V}^{\rdf}_{\moleidxone\moleidxtwo}.  
\end{align}
The effectiveness of this proposal has never been tested, and is still an open question.

\subsection{The accuracy of the number probability}

The accuracy of number probability $p(N_\AT)$ is investigated by a
Taylor expansion w.r.t.~the relative size of the atomistic region to
the whole system, which vanishes in the thermodynamic limit. The sufficient condition for the first order accuracy
asks for a balance of the chemical potential~\cite{wang2013grand}:
\begin{align}\label{eqn:mu-eq}
  \mu_\CG - \mu_\AT = \omega_0,
\end{align}
where $\omega_0$ corresponds to the work from the filter, on each
molecule that enters the atomistic region. An important conclusion
form Ref.~\cite{wang2013grand} is that the relation~\eqref{eqn:mu-eq}
holds when the thermodynamic force is applied so that the atomistic
and coarse-grained densities match. The conclusion is derived under
the assumption of thermodynamic limit and the decorrlation between the
atomistic and coarse-grained regions.

The second order accuracy is achieved if the isothermo-compressibilty of the atomistic and coarse-grained sizes matches:
\begin{align}
  \kappa_\AT = \kappa_\CG
\end{align}
This is achieved by coarse-grained modeling, for example, the
structure based coarse-grained models that reproduces the atomistic
RDF~\cite{wang2009comparative}.

\subsection{The WCA potential as a generic energy and particle reservior}

From the accuracy analysis one surprising fact is that if the
requirement for the accuracy is not very high, and only applying the
thermodynamic force is acceptable, then there is actually no
restriction on the coarse-grained model. One can even use ideal gas as
a coarse-grained model, but the computational difficulty lies in the
sudden switching on of the atomistic interaction when a coarse-grained
molecule enters the hybrid region. If two molecules are entering at
the same time and the same location, then the atomistic contribution
to the intermolecular interaction~\eqref{eqn:f-f-interpol} or
\eqref{eqn:f-v-interpol} is infinity, which drives the simulation
unstable. This numerical difficulty can be avoided by using capped
atomistic interaction~\cite{praprotnik2005adaptive}, or by using a
gradually switching-on core-softened atomistic
interaction~\cite{heyes2010thermodynamic}.
In Ref.~\cite{wang2013grand}, the authors instead tested with
Weeks-Chandler-Andersen (WCA) potential~\cite{weeks1971role} as the
coarse-grained model, which is a short-ranged and pure repulsive
interaction.  For a SPC/E water system, the cut-off radius of WCA potential
can be 2.4 time smaller than the cut-off used in the atomistic
region. Since the computational cost with is of order $\mo(r_c^3)$,
the computational cost spent on each pair interaction
is 19 times cheaper. For each pair of molecules the atomistic model computes
10 pairwise interactions (9 electrostatic + 1 van der Waals), while the
WCA model computes only one, therefore the WCA model in total costs $1/190$ 
computational effort on force computation than atomistic model.
The WCA approach has been successfully used in calculating the
chemical potential of various complex fluids and mixtures~\cite{agarwal2014chemical}.


% Therefore
% certain manipulation in the hybrid region is need improve the accuracy
% of $p(\vect x_\HY, N_\HY\vert N_\AT)$. However, it is still not clear
% what is the sufficient conditions to achieve good accuracy in $p(\vect x_\HY, N_\HY\vert N_\AT)$.
% It is easy to raise necessary conditions: The marginal distributions
% of the hybrid cooridnates
% \begin{align}
%   p(\vect x_\HY) = \sum_{N_\HY = 0}^\infty \sum_{N_\AT = 0}^\infty p(\vect x_\HY, N_\HY\vert N_\AT) p(N_\AT)
% \end{align}
% should be the same as the reference system. Here we use the fact that
% the probability $ p(N_\AT)$ should be of good accuracy, which will be
% discussed later. This approach is validated because the discussion of
% accuracy of $ p(N_\AT)$ does not require a good accuracy of $ p(\vect
% x_\HY, N_\HY\vert N_\AT)$.  The $m$th marginal distribution of a probability density saying $p(\vect x_\HY)$ is defined
% by
% \begin{align}
%   p_\HY^{(m)}(\vect x_1, \cdots \vect x_m) = \int\cdots\int d\vect x_{m+1}\cdots d\vect x_{N_\HY}
%   p (\vect x_1, \cdots, \vect x_m, \vect x_{m+1}, \cdots \vect x_{N_\HY})
% \end{align}
% In a homogeneous system, the first marginal distribution is the
% density profile in the


% It is not desirable to explicitly write the 
      
% Eq.~\eqref{eqn:dist-1-0} and \eqref{eqn:dist-1-1} are equivalent to 

% In 
% A good accuracy means that the atomistic region is embeded into the AdResS
% To answer what is an accurate equilibrium AdResS simulation,
% To understand the statistical properties of an AdResS system, 


\section{Thermodynamic properties of adaptive resolution simulation}
\label{sec:thermodynamic}

For the 
In Ref.~\cite{wang2013grand}, it has been shown that



\section{Dynamical properties: adaptive resolution simulation beyond equilibrium}
citation~\cite{wang2014exploring}

\bibliography{ref}{}
\bibliographystyle{unsrt}

% \begin{thebibliography}{}
% % and use \bibitem to create references.
% \bibitem{RefJ}
% % Format for Journal Reference
% Author, Journal \textbf{Volume}, (year) page numbers
% % Format for books
% \bibitem{RefB}
% Author, \textit{Book title} (Publisher, place year) page numbers
% % etc
% \end{thebibliography}

\end{document}

% end of file template.tex

