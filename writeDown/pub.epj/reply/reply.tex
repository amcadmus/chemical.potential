\documentclass{article}

\addtolength{\textwidth}{2.5cm}
\addtolength{\hoffset}{-.4in}

\usepackage{graphicx}
\usepackage[margin=10pt,font=small,labelfont=bf]{caption}
\usepackage{lipsum}

\renewcommand{\baselinestretch}{1.0}
% \setlength{\footskip}{0cm}
%\addtolength{\textheight}{10cm}
% \sloppy

\renewcommand{\arraystretch}{1.2}


\newcommand{\confa}[0]{{\alpha_{\footnotesize\textrm{R}}}}
\newcommand{\confb}[0]{{\textrm{C}7_{\textrm{eq}}}}
\newcommand{\confc}[0]{{\alpha_{\footnotesize\textrm{L}}}}
\newcommand{\confd}[0]{{\textrm{C}7_{\textrm{ax}}}}


\begin{document}


\hfill\begin{tabular}{@{}p{.4\linewidth}@{}}
\multicolumn{1}{@{}c@{}}{} \\
        Han Wang\\
        CAEP Software Center for \\
        High Performance Numerical\\
        Simulation\\
        Huayuan Road 6\\
        Beijing 100088, China\\
        \\
        \today
\end{tabular}
\vskip .1cm
\noindent
\begin{tabular}{@{}p{.45\linewidth}@{}}
\multicolumn{1}{@{}c@{}}{} \\
PD Dr. Luigi Delle Site \\
\\
Freie Universit\"at Berlin \\
Institute for Mathematics\\
Arnimallee 6 \\
14195, Berlin \\
  Germany\\  
\end{tabular}\hfill

\vskip 1cm

\noindent
\textbf{\large Resubmission for manuscript}

\vskip .5cm
\noindent
Dear PD Dr. Luigi Delle Site,\\

On behalf of the co-author, I would like to resubmit our article:
\textit{``Adaptive Resolution Simulation in Equilibrium and Beyond''}
by H.~Wang and A.~Agarwal to the
\textit{European Physical Journal Special Topics}.
We gratefully acknowledge the reviewers for the positive reports, and for
their insight and suggestions, which helped us improving the quality
of the manuscript.  In response to their comments, the major questions raised by the referees are
reported below in detail.  

We revised the manuscript respectively and reworked the English.
To help the referees spot the major changes in
the manuscript, we have highlighted them in red.

We hope that the revised manuscript is fit for publication now.

\vskip 1cm

\noindent Best regards,\\
Han Wang

\vskip 1cm
\newpage
\section*{Reply to Referee 1}

\noindent
\textit{For the sake of being a self-contained text, I would like some
elaboration of the claim "energy conservation is not a must for
theoretical analyses on the equilibrium statistical properties". A
statement discussed in properly quoted references, but some words may be
helpful here as well.}\\

\noindent
Following this suggestion, We have added a short comment on page 2, pointing out the result that
the force interpolation AdResS (without energy conservation) achieves very high accuracy in
equilibrium properties, such as the radial distribution function, the three-body correlation
and the chemical potential.\\


\noindent
\textit{"... although the system is in equilibrium, some properties, which are
called dynamical properties, cannot be simply computed from the
equilibrium ensemble averages". This is actually not true; in fact,
within the linear response theory, transport (dynamical) properties are
sampled via equilibrium averages of spontaneous fluctuations, as, by the
way, the authors discuss later.
The sentence should be made more accurate.
}\\

\noindent
The referee is correct, and we have changed the sentence to
``... some properties, which are called dynamical properties,
depend not only on the equilibrium ensemble, but also on the dynamics of the system.''\\

\noindent
Further, a footnote has been added to page 2 to help the readers
understand how the dynamics of system plays a role in the definition
of velocity auto-correlation, which was used as a typical example of the  dynamical properties.
\\

\noindent
\textit{In Eq. 3, the authors describe a potential interpolation scheme which
is different from the one presented in e.g., Ref 21 in this very
manuscript. There is no contradiction in the way the potential
interpolation scheme is presented in the present manuscript, but the
reader may find helpful to learn that there are different
potential-based adaptive resolution schemes. For the sake of brevity,
this discussion could/should be limited to few sentences.}
\\

\noindent
A footnote has been added to page 4, discussing the difference between the potential interpolation
used in this manuscript and that of Ref.21.\\

\noindent
\textit{At the beginning of paragraph 3, the authors write two times that the
"the subsystem/atomistic region is subject to the grand-canonical
ensemble". I think "subject" is not the right word. A (sub)system may
sample the ensemble, not being subject to.
}\\

\noindent
Corrected. We have changed ``subject'' to ``sample''.\\

\noindent
\textit{$\beta$ ( = 1/kT) should be defined E(E.g, after Eq. 9). It is standard
to use the letter $\beta$, but it is customary to define it.}\\

\noindent
The definition has been added under Eq.9.\\

\noindent
\textit{IMPORTANT: In Fig. 5 the discrepancy between all-atomistic and AdResS
schemes increases the longer the time-scale related to an eigenvalue of
the MSM transition matrix is. It is true the agreement between the two
schemes is satisfactory within the error bars, but is this increased
difference for increasing time scale a systematic behaviour or is it
limited to the particular system studied? In either case, why this
happens? Is there a strategy to improve the accuracy of such longer
time-scales in case they are important to be known with high accuracy?
Even though the answers to all these questions are not all known, some
discussion along these lines may be helpful to the reader.}\\

\noindent
The leading time-scale is usually estimated with higher statistical
uncertainty, because the corresponding transition is less frequently
sampled in an MD simulation than other transition events, owing to its longer implied time-scale.
Moreover, the fact that the probability of being in states $\{\confc,\confd\}$ is
much less that that of being in state $\{\confa \}$ also leads to rare transition.
This can be seen by,
e.g., considering a simplified 2-state (saying $A$ and $B$) reversible Markov model:
if $p(A) \gg p(B)$, by $ p(B\vert A) p(A) = p(A\vert B) p(B) $, we have $p(B\vert A) \ll p(B\vert A) \leq 1$, which indicates that the 
transition from $A$ to $B$ is rare.\\

\noindent
Longer MD simulations will improve the accuracy.  Deliberately choosing
initial states for MD runs also helped in recovering leading
time-scales that are even longer than one single MD simulation, see,
e.g., Kohlhoff et. al. Nat. Chem. 1, 15--21, 2014.  Discussion along
these lines have been added to page 16.


\section*{Reply to Referee 2}

\noindent
\textit{In the beginning of the paper (page 3) the paper reads ``AdResS
requires all interactions being treated by the cutoff method''
find the statement a bit vague. It is not clear if this is a specific
cut-off method that is used in AdResS, and in that case should be
defined or a reference should be added, or is just applying a general
cutoff, and in that case the sentence should be slightly modified to
something along the lines of ``a cut-off has to be applied''.}\\

\noindent
The cut-off method is the same as the general cut-off method  used in normal MD simulations.
The electrostatic interactions are treated by the Reaction-Field method, so they are also cut-off.
The footnote 2 on page 3 has been revised, and a new footnote has been added on page 3.


\end{document}
