%%%%%%%%%%%%%%%%%%%%%%% file template.tex %%%%%%%%%%%%%%%%%%%%%%%%%
%
% This is a template file for The European Physical Journal Special Topics
%
% Copy it to a new file with a new name and use it as the basis
% for your article
%
%%%%%%%%%%%%%%%%%%%%%%%% Springer-Verlag %%%%%%%%%%%%%%%%%%%%%%%%%%
%
\documentclass[epjST]{svjour}
%
\usepackage{graphics}
\usepackage{color, colortbl}
\usepackage{amsmath,amssymb}
\usepackage{graphicx}
\usepackage{tabularx}

\newcommand{\recheck}[1]{{\color{red} #1}}
\newcommand{\redc}[1]{{\color{red} #1}}
\newcommand{\bluec}[1]{{\color{red} #1}}
\newcommand{\greenc}[1]{{\color{green} #1}}
\newcommand{\vect}[1]{\textbf{\textit{#1}}}
\newcommand{\dd}[1]{\textsf{#1}}
\newcommand{\fwd}[0]{\textrm{fw}}
\newcommand{\bwd}[0]{\textrm{bw}}
\newcommand{\period}[0]{T_{\textrm{P}}}
\newcommand{\ml}[0]{\mathcal {L}}
\newcommand{\mo}[0]{\mathcal {O}}
\newcommand{\mbp}[0]{\mathbb {P}}
\newcommand{\mh}[0]{\mathcal {H}}
\newcommand{\dist}[0]{\textrm {dist}}
\newcommand{\AT}[0]{\textrm{AT}}
\newcommand{\HY}[0]{\textrm{HY}}
\newcommand{\CG}[0]{\textrm{CG}}
\newcommand{\EX}[0]{\textrm{EX}}
\newcommand{\moleidxone}[0]{i}
\newcommand{\moleidxtwo}[0]{j}
\newcommand{\atomidxone}[0]{\alpha}
\newcommand{\atomidxtwo}[0]{\beta}
\newcommand{\thf}{{\textrm{th}}}
\newcommand{\rdf}{{\textrm{rdf}}}
\newcommand{\rep}{{\textrm{rep}}}
\newcommand{\dof}{{\textrm{DOF}}}
\newcommand{\exc}{{\textrm{extra}}}
\newcommand{\equi}{{\textrm{eq}}}

\newcommand{\confa}[0]{{\alpha_{\textrm{R}}}}
\newcommand{\confb}[0]{{\textrm{C}7_{\textrm{eq}}}}
\newcommand{\confc}[0]{{\alpha_{\textrm{L}}}}
\newcommand{\confd}[0]{{\textrm{C}7_{\textrm{ax}}}}


% \usepackage{amsfonts}
% \newcommand{\tickYes}{\checkmark}
% \usepackage{pifont}
% \newcommand{\tickNo}{\hspace{1pt}\ding{55}}
% \definecolor{MyGray}{gray}{0.85}

%
\begin{document}
%
\title{Reply to comment by R.~Klein on ``Adaptive Resolution Simulation in Equilibrium and Beyond''}
\author{Han Wang\inst{1}\fnmsep\thanks{\email{wang\_han@iapcm.ac.cn}}, Animesh Agarwal\inst{2}}
%
\institute{CAEP Software Center for High Performance Numerical Simulation, Beijing, China
  \and Institut f\"ur Mathematik, Freie Universit\"at Berlin, Berlin, Germany}
%
% \abstract{
%   We reply comments~\cite{klein2015comment}
%   on our recent paper~\cite{wang2015adaptive}.
% } %end of abstract
%
\maketitle
%

We would like to thank R.~Klein for providing valuable comments~\cite{klein2015comment}
on our recent paper~\cite{wang2015adaptive}. 
The questions raised therein will be replied in the following. 

\paragraph{Particles in the hybrid region}
In the AdResS Scheme, the minimum thickness of the hybrid region
should be $r_c$ (cut-off radius) if the traditional weighting function
(i.e.~Eq.1 in \cite{wang2015adaptive}) is used, while it should be at least
$2r_c$ if a buffer region  is added to the hybrid
region (i.e.~Eq.8 in the same reference).
The average molecular spacing of liquid water under ambient
condition is about 0.3~nm. Therefore, if one takes 1.2~nm as the cut-off
radius, the thickness is 4 and 8 times the
the average molecular spacing for traditional and buffered hybrid region, respectively.

We assume that the interactions between particles are pairwise, and denote
the energy contribution of one particle pair by $u(r)$, with $r$ being the distance
between the particles. Cutting-off this interaction means that it is assumed to vanish
if the distance between the particles is larger than the cut-off radius $r_c$.
``The correlation between the atomistic and the coarse-grained
regions is negligible'' means that the
degrees of freedom (positions and velocities) of two particles
in the atomistic and the coarse-grained regions, respectively, are almost un-correlated.

It is true that by introducing the adaptive resolution, the system is not homogeneous
and isotropic anymore. Therefore, strictly speaking, the standard definition
of RDF's and 3-body correlation functions is problematic. However, if the configuration
of the coarse-grained and the
hybrid regions are designed with high accuracy, one can approximately
treat the system as homogeneous and isotropic, and the RDF and 3-body correlation
are well-defined for (at least) the center-of-mass configuration of water system.
On the other hand, if the hybrid
configuration deviates from the full-atomistic reference,
one can anyway compute the RDF,
which is the averaged anisotropic structure around a particle, and contains information on
the difference to the
full-atomistic reference (probably with positive-negative cancellation).
In practice,
it has been shown that this information is sufficient for computing
a RDF correlation that drives the system
to the correct RDF in the hybrid region~\cite{wang2012adaptive,wang2013grand}.

% it has been shown that  
% the RDF correlation can be effectively computed by using
% this averaged RDF.


\paragraph{The thermodynamic force}
The thermodynamic force in the atomistic and the coarse-grained
regions are forced to vanish, that means if a zero thermodynamic force
is used initially, then only the value in the hybrid region is updated
iteratively. In order to calculate the density,
take the pure TIP3P water system in Ref~\cite{wang2015adaptive}
for example, we did the following:
\begin{enumerate}
\item The hybrid region was divided into bins of width 0.01~nm in the
  direction of changing resolution.
\item The number of molecules (center-of-mass position) in
  each bin was counted and averaged to calculate a density. Each density data point
  has a statistical uncertainty of roughly 8\%.
  The productive trajectory for the computation was 200~ps long
  and the molecular positions were saved every 0.2~ps.
\item The resulting density profile was still oscillatory,
  so we smooth it by fitting the density
  to a  spline with a grid spacing of 0.4~nm.
  The spline was calculated by VOTCA~\cite{ruehle2009versatile} software.
\item This smoothed density was used
  by the iterative algorithm (Eq.~21 in Ref.~\cite{wang2015adaptive}),
  % that computes the
  and the thermodynamic force converged within 10 iterations.
\end{enumerate}


\paragraph{Particular properties of the SPC/E water system}
Actually we used AdResS to compute the chemical
potential for various liquids and mixtures in
Ref.~\cite{agarwal2014chemical}. In the simulations, only 
thermodynamic force was calculated and applied to correct the density
in the hybrid region. The chemical potential was computed with
satisfactory accuracy.  Moreover, although no RDF correction was used
and the system was simulated under extreme technical conditions (very
small atomistic and coarse-grained regions and a relatively large
hybrid region), the RDFs were reproduced with satisfactory
agreement to full-atomistic reference (see e.g.~Fig.3 of Ref.~\cite{agarwal2014chemical}).

\paragraph{Structure of the hybrid zone}
It would be very interesting to establish an analogue between
a phase coexisting system
and an AdResS system.
We would like to note that the major difference between them
is that the former satisfies both the momentum and energy
conservations, while the latter can satisfy only one of the two.
This difference may play a significant role in deriving the final conclusions.



\bibliography{ref}{}
\bibliographystyle{unsrt}

% \begin{thebibliography}{}
% % and use \bibitem to create references.
% \bibitem{RefJ}
% % Format for Journal Reference
% Author, Journal \textbf{Volume}, (year) page numbers
% % Format for books
% \bibitem{RefB}
% Author, \textit{Book title} (Publisher, place year) page numbers
% % etc
% \end{thebibliography}

\end{document}

% end of file template.tex

